\chapter{Ma�-Theorie}
Es sei $\Omega$ eine vorgegebene Menge. 
Unter einem \emph{System von Mengen auf $\Omega$} verstehen wir eine beliebige
Teilmenge der Potenzmenge von $\Omega$. 

\begin{Definition}[$\sigma$-Algebra]
  Gegeben sei eine Menge $\Omega$.  Es $\Sigma$ ein System von Mengen auf
  $\Omega$, also $\Sigma \subseteq 2^\Omega$.  Dann nennen wir $\Sigma$ eine 
  \emph{$\sigma$-Algebra auf $\Omega$} falls gilt:
  \begin{enumerate}
  \item $\Omega \in \Sigma$.
  \item $\forall A \in \Sigma: A^c \in \Sigma$.

        Hier bezeichnet $A^c := \Omega \backslash A$ das \emph{Komplement} von $A$
        in der Menge $\Omega$.
  \item F�r jede Folge $\folge{A_n}$ von Elementen aus $\Sigma$ ist auch die
        Vereinigung aller Folgenglieder ein Element aus $\Sigma$: 
        \\[0.1cm]
        \hspace*{1.3cm}
        $\bigl(\forall n \in \mathbb{N}: A_n \in \Sigma\bigr) \rightarrow
         \bigcup\limits_{n\in \mathbb{N}} A_n \in \Sigma$.
         \hspace*{\fill} $\Box$
  \end{enumerate}
\end{Definition}
$\sigma$-Algebren sind also Teilmengen der Potenzmenge einer gegebenen Menge,
die die Menge $\Omega$ enthalten und abgeschlossen sind unter Komplementbildung
und abz�hlbarer Vereinigung. 

\begin{Definition}[messbarer Raum]
  Wir bezeichnen ein Paar $\pair(\Omega, \Sigma)$ als \emph{messbaren Raum} 
  falls $\Sigma$ eine $\sigma$-Algebra auf $\Omega$ ist.  In diesem Fall hei�en
  die Elemente aus $\Sigma$ \emph{messbare Mengen}.
  \qed
\end{Definition}

\begin{Definition}[Topologie]
  Gegeben sei eine Menge $\Omega$.  Es $\mathcal{O}$ ein System von Mengen auf 
  $\Omega$, also $\mathcal{O} \subseteq 2^\Omega$.  Dann nennen wir $\mathcal{O}$ eine
  \emph{Topologie $\Omega$} falls gilt:
  \begin{enumerate}
  \item $\emptyset \in \mathcal{O}$.
  \item $\Omega \in \mathcal{O}$.
  \item $\forall A,B \in \mathcal{O}: A \cap B \in \mathcal{O}$.
  \item F�r jede Menge $\bigl\{ A_i | i\in I\bigr\}$ von Elementen aus $\mathcal{O}$ ist auch die
        Vereinigung dieser Menge ein Element aus $\mathcal{O}$: 
        \\[0.1cm]
        \hspace*{1.3cm}
        $\bigl(\forall i \in i: A_i \in \mathcal{O}\bigr) \rightarrow
         \bigcup\limits_{i \in I} A_i \in \mathcal{O}$.
         \hspace*{\fill} $\Box$
  \end{enumerate}
\end{Definition}
Eine Topologie $\mathcal{O}$ ist also eine Teilmengen der Potenzmenge einer gegebenen Menge,
die sowohl die leere Menge  als auch die Menge $\Omega$ enth�lt und die
abgeschlossen ist unter der Bildung des Schnittes zweier Mengen und
und der beliebigen Vereinigung von Mengen aus $\mathcal{O}$. 

\begin{Definition}[topologischer Raum]
  Wir bezeichnen ein Paar $\pair(\Omega, \mathcal{O})$ als \emph{topologischen Raum} 
  falls $\mathcal{O}$ eine Topologie auf $\Omega$ ist.  In diesem Fall hei�en
  die Elemente aus $\mathcal{O}$ \emph{offene Mengen}.
  \qed
\end{Definition}

\begin{Definition}[messbare Abbildung]  Es sei 
  \begin{enumerate}
  \item $\pair(X,\Sigma)$ ein messbarer Raum
  \item $\pair(Y,\mathcal{O})$ ein topologischer Raum
  \item $f:X \rightarrow Y$ eine Abbildung.
  \end{enumerate}
  Dann nennen wir die Abbildung $f$ \emph{messbar} falls die Urbilder aller
  offenen Mengen messbar sind: 
  \\[0.1cm]
  \hspace*{1.3cm}
  $\forall U \in \Sigma : f^{-1}(U) \in \mathcal{O}$.
  \qed
\end{Definition}

\begin{Definition}[erzeugte $\sigma$-Algebra]
  Es sei $\Omega$ eine Menge und $\mathfrak{F}$ ein System von Mengen auf
  $\Omega$.  Dann definieren wir die von $\mathfrak{F}$ erzeugte
  $\sigma$-Algebra $\sigma(\mathfrak{F})$ als die kleinste $\sigma$-Algebra,
  die $\mathfrak{F}$ enth�lt: 
  \\[0.1cm]
  \hspace*{1.3cm}
  $\sigma(\mathfrak{F}) := \bigcap \bigl\{ \Sigma \subseteq 2^\Omega \;\big|\; \mbox{$\Sigma$ ist $\sigma$-Algebra auf $\Omega$} \bigr\}$.
  \qed
\end{Definition}

\begin{Definition}[Borel'sche $\sigma$-Algebra, $\mathfrak{B}(\mathcal{O})$]
  Es sei $\pair(\Omega, \mathcal{O})$ ein topologischer Raum.  Dann wird von 
  der Topologie $\mathcal{O}$ eine $\sigma$-Algebra auf $\Omega$ erzeugt,
  die wir die \emph{Borel'sche $\sigma$-Algebra} nennen und mit
  $\mathfrak{B}(\mathcal{O})$ bezeichnen: 
  \\[0.1cm]
  \hspace*{1.3cm}
  $\mathfrak{B}(\mathcal{O}) := \sigma(\mathcal{O})$.
  \\[0.1cm]
  Die Elemente aus $\mathfrak{B}(\mathcal{O})$ bezeichnen wir als
  \emph{Borel'sche Mengen}.
  \qed
\end{Definition}

\begin{Satz} \label{satz1} Es gelte:
  \begin{enumerate}
  \item $\pair(X,\Sigma)$ ist ein messbarer Raum.
  \item $f:X \rightarrow Y$ bildet $X$ in $Y$ ab.
  \item $\Omega = \bigl\{ E \in 2^Y \;\big|\; f^{-1}(E)\in\Sigma \bigr\}$
  \end{enumerate}
  Dann ist $\Omega$ eine $\sigma$-Algebra auf $Y$.
\end{Satz}

\noindent
\textbf{Beweis}: Wir m�ssen zeigen, dass $\Omega$ die bei der Definition
der $\Sigma$-Algebra geforderten Abschluss-Eigenschaften hat.
\begin{enumerate}
\item Wir zeigen $Y \in \Omega$: 

      Nach Definition von $\Omega$ gilt 
      \\[0.1cm]
      \hspace*{1.3cm}
      $Y \in \Omega$ \quad g.d.w. \quad $f^{-1}(Y) \in \Sigma$.
      \\[0.1cm]
      Elementare Mengenlehre zeigt $f^{-1}(Y) = X$. 
      Da $\Sigma$ eine $\sigma$-Algebra auf $X$ ist, gilt $X \in \Sigma$,
      also $f^{-1}(Y) \in \Sigma$ und damit $Y \in \Omega$.
\item Es sei $E \in \Omega$.  Wir m�ssen $E^c \in \Omega$ zeigen. Es gilt: 
      \\[0.3cm]
      \hspace*{1.3cm}
      $
      \begin{array}[t]{lcl}
       E^c \in \Omega & \leftrightarrow & f^{-1}(Y \backslash E) \in \Sigma \qquad \mbox{nach Definition von $\Omega$} \\[0.3cm]
                      & \leftrightarrow & f^{-1}(Y) \backslash f^{-1}(E) \in \Sigma \\[0.1cm]
                      & & \mbox{denn f�r beliebige Mengen $A$ und $B$ gilt:} \\[0.1cm]
                      & & \qquad f^{-1}(A \backslash B) = f^{-1}(A) \backslash f^{-1}(B) \\[0.3cm]
                      & \leftrightarrow & X \backslash f^{-1}(E) \in \Sigma \qquad \mbox{denn $f^{-1}(Y) = X$} \\[0.3cm]
                      & \leftrightarrow & f^{-1}(E) \in \Sigma \qquad \\[0.1cm]
                      &&   \mbox{denn $\Sigma$ ist abgeschlossen unter Komplementbildung} \\[0.3cm]
                      & \leftrightarrow & E \in \Omega \\[0.3cm]
                      & \leftrightarrow & \mathtt{true} \\[0.3cm]
      \end{array}
      $
\item Es sei eine Folge $\folge{E_n}$ mit $E_n \in \Omega$ f�r alle $n \in \mathbb{N}$
      gegeben.  F�r alle $n \in \mathbb{N}$ gilt also $f^{-1}(E_n) \in \Sigma$.
      Zu zeigen ist dann $\bigcup\limits_{n\in \mathbb{N}} E_n \in \Omega$. 
      Es gilt: 
      \\[0.3cm]
      \hspace*{1.3cm}
      $
      \begin{array}[t]{lcl}
        \bigcup\limits_{n\in \mathbb{N}} E_n \in \Omega & \leftrightarrow &
        f^{-1}\left(\bigcup\limits_{n\in \mathbb{N}} E_n\right) \in \Sigma 
        \qquad \mbox{nach Definition von $\Omega$} \\[0.5cm]
        & \leftrightarrow & \bigcup\limits_{n\in \mathbb{N}} f^{-1}(E_n) \in \Sigma \\[0.5cm]
          && \mbox{denn f�r beliebige Systeme von Mengen $\bigl(A_i\bigr)_{i\in I}$ gilt} \\ 
          && \qquad f^{-1}\left(\bigcup\limits_{i\in I} A_i\right) = \bigcup\limits_{i\in I} f^{-1}(A_i) \\[0.3cm]
        & \leftrightarrow & \mathtt{true} \\[0.3cm]
        && \mbox{denn nach Voraussetzung gilt $f^{-1}(E_n) \in \Sigma$ und} \\
        &&   \mbox{$\Sigma$ ist abgeschlossen unter abz�hlbarer Vereinigung.}
      \end{array}
      $\qed
\end{enumerate}



\begin{Satz} Es gelte:
  \begin{enumerate}
  \item $\pair(X,\Sigma)$ ist ein messbarer Raum.
  \item $\pair(Y,\mathcal{O})$ ist ein topologischer Raum.
  \item $f:X \rightarrow Y$ ist messbar.
  \end{enumerate}
  Dann  sind die Urbilder der Borel'schen Mengen $\mathfrak{B}(\mathcal{O})$ messbar.  
\end{Satz}

\noindent
\textbf{Beweis:} Wir definieren die Menge $\Omega$ wie in Satz \ref{satz1}: 
\\[0.1cm]
\hspace*{1.3cm} $\Omega := \bigl\{ E \subseteq Y \;\big|\; f^{-1}(E) \in \Sigma \bigr\}$
\\[0.1cm]
Ist $A \in \mathcal{O}$, dann folgt aus der Messbarkeit von $f$, dass $
f^{-1}(A) \in\Sigma$ gilt.  Also haben wir 
\\[0.1cm]
\hspace*{1.3cm} $\mathcal{O} \subseteq \Omega$.
\\[0.1cm]
Weil $\Omega$ nach Satz \ref{satz1} eine $\sigma$-Algebra ist, folgt 
\\[0.1cm]
\hspace*{1.3cm} $\mathfrak{B}(\mathcal{O}) \subseteq \Omega$.
\\[0.1cm]
Also gilt f�r alle $E \in \mathfrak{B}(\mathcal{O})$  
\\[0.1cm]
\hspace*{1.3cm} $f^{-1}(E) \in \Sigma$. \qed
\vspace*{0.3cm}

\noindent
Wir definieren  $\overline{\mathbb{R}} := \mathbb{R} \cup\{-\infty,+\infty\}$

\begin{Satz} Es gelte:
  \begin{enumerate}
  \item $\pair(X,\Sigma)$ ist ein messbarer Raum.
  \item $f:X \rightarrow \overline{\mathbb{R}}$ sei eine numerische Funktion.
  \item $\forall\alpha \in \mathbb{R} : f^{-1}\bigl( (\alpha,\infty] \bigr) \in \Sigma$.
  \end{enumerate}
  Dann ist $f$ messbar.
\end{Satz}

\noindent
\textbf{Beweis}: Wir definieren 
\\[0.1cm]
\hspace*{1.3cm}
$\Omega:= \bigl\{ E \subseteq \overline{\mathbb{R}} \;\big|\; f^{-1}(E) \in \Sigma \bigr\}$.
\\[0.1cm]
Nach Satz \ref{satz1} ist $\Omega$ eine $\sigma$-Algebra.
Nach Voraussetzung sind zun�chst alle Intervalle der Form $(\alpha,\infty]$ in $\Omega$.
Wegen 
\begin{enumerate}
\item $[-\infty,\alpha] = \overline{\mathbb{R}} \backslash (\alpha,+\infty]$, 
\item $[-\infty,\alpha) = \bigcup\limits_{n=1}^\infty [-\infty,\alpha - \frac{1}{n}]$,
\item $(\alpha,\beta)   = [-\infty,\beta) \cap (\alpha,\infty]$, 
\end{enumerate}
folgt aus der Abgeschlossenheit der $\sigma$-Algebra $\Sigma$ unter der Bildung
von Komplementen und Schnitten, dass offene Intervalle der Form $(\alpha,\beta)$ in 
$\Omega$ liegen.

%%% Local Variables: 
%%% mode: latex
%%% TeX-master: "statistik.tex"
%%% End: 
