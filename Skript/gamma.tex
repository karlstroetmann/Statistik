\begin{Definition}[Gamma-Funktion]
  F�r alle $z \in \mathbb{R} \backslash \bigl\{ -n \;\big|\; n \in \mathbb{N} \bigr\}$
  definieren wir die Gamma-Funktion durch
  \\[0.1cm]
  \hspace*{1.3cm}
  $\displaystyle \Gamma(z) := \int_0^\infty x^{z-1} e^{-x} dx$.  
\end{Definition}
Die Gamma-Funktion ist eine Fortsetzung der Falkult�t auf die positiven reellen Zahlen,
denn es gilt:
\\[0.1cm]
\hspace*{1.3cm}
$\Gamma(n+1) = n!$
\\[0.1cm]
\textbf{Beweis:}  Wir f�hren den Beweis durch Induktion nach der Zahl $n$.
\begin{enumerate}
\item[I.A.:] $n=0$.  Es gilt 
             \\[0.1cm]
             \hspace*{1.3cm}
             $
             \begin{array}[t]{lcl}
               \Gamma(1) 
               & = & \displaystyle \int_0^\infty x^{1-1} e^{-x}\, dx \\[0.3cm]
               & = & \displaystyle \int_0^\infty e^{-x}\, dx \\[0.3cm]
               & = & \displaystyle -e^{-x} \big|_0^\infty  \\[0.3cm]
               & = & \displaystyle - e^{-\infty} + e^0  \\
               & = & 1 \\
               & = & 0!
             \end{array}
             $
\item[I.S.:] $n \mapsto n+1$.  Es gilt 
             \\[0.1cm]
             \hspace*{1.3cm}
             $
             \begin{array}[t]{lcl}
               \Gamma\bigl((n+1)+1\bigr) 
               & = & \displaystyle \int_0^\infty x^{(n+2)-1} e^{-x}\, dx \\[0.3cm]
               & = & \displaystyle \int_0^\infty x^{n+1} e^{-x}\, dx \\[0.3cm]
               &   & \mbox{partielle Integration mit $u(x) = x^{n+1}$, $v'(x) = e^{-x}$,} \\
               &   & \mbox{also $u'(x) = (n+1)\cdot x^n$, $v(x) = -e^{-x}$ liefert:}      \\[0.3cm]
               & = & \displaystyle \underbrace{-x^{n+1}\cdot e^{-x}\big|_0^\infty}_0 + \int_0^\infty (n+1)\cdot x^{n} e^{-x}\, dx \\[0.3cm]
               & = & \displaystyle (n+1)\cdot \int_0^\infty x^{(n+1)-1} e^{-x}\, dx \\[0.3cm]
               & = & \displaystyle (n+1)\cdot \Gamma(n+1) \\[0.3cm]
               & \stackrel{IV}{=} & \displaystyle (n+1)\cdot n! \\
               & = & \displaystyle (n+1)! \hspace*{\fill} \Box
             \end{array}
             $
\end{enumerate}
Es w�re intuitiver, wenn $\Gamma(z)$ als $\int_0^\infty x^{z} e^{-x}\, dx$ definiert w�re,
denn dann w�rde $\Gamma(n) = n!$ gelten.  Ais historischen Gr�nden ist die Definition aber
komplizierter.

%%% Local Variables: 
%%% mode: latex
%%% TeX-master: "statistik"
%%% End: 
