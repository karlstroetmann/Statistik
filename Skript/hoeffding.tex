\chapter{Die Ungleichung von Hoeffding}
Angenommen, wir haben eine Urne mit wei�en und schwarzen Kugeln.  Die Wahrscheinlichkeit, dass eine
beliebige Kugel wei� ist, habe den Wert $\mu$, der uns allerdings nicht bekannt ist.  Wir entnehmen
der Urne eine Stichprobe vom Umfang $N$ mit zur�cklegen, so dass die Wahrscheinlichkeit, dass ein
bestimmte Kugel wei� ist, immer den Wert $\mu$ hat.  Ziehen wir bei der Stichprobe insgesamt $k$
wei�e Kugeln, so k�nnen wir die Wahrscheinlichkeit, dass der Bruchteil $k/N$ um mehr als
$\varepsilon$ von der Wahrscheinlichkeit $\mu$ abweicht, durch die
\href{https://en.wikipedia.org/wiki/Hoeffding%27s_inequality}{Hoeffding Ungleichung}
(\href{https://de.wikipedia.org/wiki/Wassily_Hoeffding} Wassily Hoeffding; 1914--1991)
\\[0.2cm]
\hspace*{1.3cm}
$\ds P\biggl(\Bigl|\frac{k}{N} - \mu\Bigr| > \varepsilon\biggr) \leq 2 \cdot\exp\bigl(-2\cdot\varepsilon^2\cdot N\bigr) $
\\[0.2cm] 
absch�tzen.  Diese Formel k�nnen wir nach $N$ umstellen.  Definieren wir $p$ als die
Wahrscheinlichkeit, dass der Anteil der wei�en Kugeln in der Stichprobe um mehr als $\varepsilon$
von der Wahrscheinlichkeit $\mu$ abweicht, setzen wir also
\\[0.2cm]
\hspace*{1.3cm}
$\ds p := P\biggl(\Bigl|\frac{k}{N} - \mu\Bigr| > \varepsilon\biggr)$,
\\[0.2cm]
so haben wir
\\[0.2cm]
\hspace*{1.3cm}
$\ds p \leq 2 \cdot\exp\bigl(-2\cdot\varepsilon^2\cdot N\bigr)$.
\\[0.2cm]
Diese Ungleichung k�nnen wir nach $N$ aufl�sen.  Wir erhalten
\\[0.2cm]
\hspace*{1.3cm}
$\ds N \geq -\frac{\ln(p/2)}{2 \cdot \varepsilon^2}$.
\\[0.2cm]
Nehmen wir beispielsweise an, dass wir einen W�rfel haben und wir sollen die Wahrscheinlichkeit
daf�r, dass eine $6$ gew�rfelt wird, mit einer Genauigkeit von $0.01$ bestimmen.  Die
Wahrscheinlichkeit $p$, dass unser Ergebnis ein Genauigkeit von $0.01$ hat, soll $95\%$ ein.  Dann gilt
\\[0.2cm]
\hspace*{1.3cm}
$p := 0.05$ \quad und \quad $\varepsilon := 0.01$.
\\[0.2cm]
Setzen wir diese Werte in die obige Formel ein, so erhalten wir
\\[0.2cm]
\hspace*{1.3cm}
$N \geq 18444.4$
\\[0.2cm]
und stellen fest, dass wir mindesten $18\,445$ mal w�rfeln m�ssen, um die Wahrscheinlichkeit, dass
eine $6$ gew�rfelt wird mit der erforderlichen Genauigkeit von $0.01$ und $95\%$ Sicherheit bestimmen zu k�nnen.

%%% Local Variables: 
%%% mode: latex
%%% TeX-master: "statistik"
%%% ispell-local-dictionary: "deutsch8"
%%% eval: (setenv "LANG" "de_DE.UTF-8")
%%% End: 
